\documentclass[12pt,titlepage,french]{article}
\usepackage{babel}
\usepackage{graphicx}
\usepackage[margin=2.5cm]{geometry}

\usepackage[hidelinks]{hyperref}
\usepackage{tabularx}
\usepackage[utf8]{inputenc}
\usepackage[T1]{fontenc}
\pagestyle{plain}

\usepackage{booktabs,makecell,tabu}
\renewcommand\theadfont{\bfseries}

\linespread{1.5}

\begin{document}
%\renewcommand{\thesection}{\arabic{section}} % utilisé pour spécifier la numérotation des sections

\begin{titlepage}
\newcommand{\HRule}{\rule{\linewidth}{0.5mm}}
\center

  \includegraphics[width=0.45\textwidth]{../ressources/img_logos/logo_polytech.png}\\[1cm]
   
  \includegraphics[width=0.45\textwidth]{../ressources/img_logos/logo_taglabs.png}


\HRule \\[0.4cm]
{ \huge \bfseries Rapport hebdomadaire\\[0.15cm] }
Classification colorimétrique de nuages de points 3D\\
Semaine du 25/11 au 01/12
\HRule \\[1.5cm]
Ronan Collier,
Mathieu Letrone,
Tri-Thien Truong
\\[1cm]
\end{titlepage}

\section{Travail effectué}

\noindent\begin{tabularx}{17cm}{|p{2.5cm}|p{2.5cm}|p{1cm}|p{1.5cm}|p{2.5cm}|X|}
    \hline
    \textbf{Tâche} & \textbf{Nature} & \textbf{Qui} & \textbf{Temps} & \textbf{Réalisation} & \textbf{Commentaire} \\
    \hline
    Amélioration et rendu du CdC & Production & ML, TTT, RC & 2 h & Fait & Les remarques de M.Guédon ont été prises en compte\\
    \hline
    Recherches sur le choix de la technologie qui sera utilisée (Python/C++) & recherche & TT & 2 h & En cours & D'autres mémoires/documents sont présents sur internet sur la segmentation/extraction d'éléments dans un nuage de points, qui donnent beaucoup d'informations sur les technologies utilisées. (voir fin du tableau pour les sources)\\
    \hline
\end{tabularx} \\

Sources (textes cliquables): 
\begin{itemize}
\item \href{https://orasis2017.sciencesconf.org/135580/document}{\textbf{Segmentation de nuages de points 3D pour le phenotypage de tournesols}, de William Gélard, Ariane Herbulot, Michel Devy, Philippe Burger}
\item \href{https://dumas.ccsd.cnrs.fr/dumas-01164570/document?fbclid=IwAR078AKyDp0CPfwM0DWix35VVFcN1GtHBQ92EN-wLI7GQrexncCMDqfIoHQ}{\textbf{Extraction d’éléments géométriques dans un nuage de points LiDAR terrestre : application aux relevés de façades}, de Mounir Ait Mansour}
\end{itemize}



\section{Prévision des tâches pour la suite de l'itération par ordre de réalisation :}

\noindent\begin{tabularx}{17cm}{|p{2.5cm}|p{2.5cm}|p{1cm}|p{1.5cm}|p{2.5cm}|X|}
    \hline
    \textbf{Tâche} & \textbf{Nature} & \textbf{Qui} & \textbf{Temps} & \textbf{Réalisation} & \textbf{Commentaire} \\
    \hline
    Avancer sur les recherches & Recherche & ML, TT, RC & 2 h & En cours & \\
    \hline
    Faire des tests avec les technologies possibles à utiliser, puis les comparer & Formation/ Production & ML, TTT, RC & 4 h & En cours &  \\
    \hline
\end{tabularx} \\

\section{Nouvelles complémentaires}

\noindent\begin{tabularx}{17cm}{|p{5cm}|X|}
\hline
   \begin{minipage}{.3\textwidth}
      \includegraphics[width=\linewidth]{img/smhappy.jpg}
    \end{minipage}
    & 
    Les choses positives de la semaine :\\
\hline
     \begin{minipage}{.3\textwidth}
      \includegraphics[width=\linewidth]{img/smneutral.jpg}
    \end{minipage}
    &
    Les inquiètudes de la semaine : \\
\hline
     \begin{minipage}{.3\textwidth}
      \includegraphics[width=\linewidth]{img/smsad.jpg}
    \end{minipage}
    &
    Les problèmes importants de la semaine : Semaine assez chargée dû aux révisions pour le devoir surveillé de Cryptographie, et des préparations pour l'événement de la Nantarena (membres du club Rézo)\\
\hline
\end{tabularx}

\end{document}
