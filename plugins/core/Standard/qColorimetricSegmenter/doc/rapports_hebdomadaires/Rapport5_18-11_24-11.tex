\documentclass[12pt,titlepage,french]{article}
\usepackage{babel}
\usepackage{graphicx}
\usepackage[margin=2.5cm]{geometry}

\usepackage[hidelinks]{hyperref}
\usepackage{tabularx}
\usepackage[utf8]{inputenc}
\usepackage[T1]{fontenc}
\pagestyle{plain}

\usepackage{booktabs,makecell,tabu}
\renewcommand\theadfont{\bfseries}

\linespread{1.5}

\begin{document}
%\renewcommand{\thesection}{\arabic{section}} % utilisé pour spécifier la numérotation des sections

\begin{titlepage}
\newcommand{\HRule}{\rule{\linewidth}{0.5mm}}
\center

  \includegraphics[width=0.45\textwidth]{../ressources/img_logos/logo_polytech.png}\\[1cm]
   
  \includegraphics[width=0.45\textwidth]{../ressources/img_logos/logo_taglabs.png}


\HRule \\[0.4cm]
{ \huge \bfseries Rapport hebdomadaire\\[0.15cm] }
Classification colorimétrique de nuages de points 3D\\
Semaine du 18/11 au 24/11
\HRule \\[1.5cm]
Ronan Collier,
Mathieu Letrone,
Tri-Thien Truong
\\[1cm]
\end{titlepage}

\section{Travail effectué}

\noindent\begin{tabularx}{17cm}{|p{2.5cm}|p{2.5cm}|p{1cm}|p{1.5cm}|p{2.5cm}|X|}
    \hline
    \textbf{Tâche} & \textbf{Nature} & \textbf{Qui} & \textbf{Temps} & \textbf{Réalisation} & \textbf{Commentaire} \\
    \hline
    Préparation soutenance & Concertation / Production & ML, TTT, RC & 3 h & Fait & Entrainements, corrections et ajustements\\
    \hline
    Soutenance & production & ML, TT, RC & 1 h & Fait & Passage de la première soutenance\\
    \hline
    Modification CdC & production & ML, TT, RC & 3 h & En cours & Correction du CdC en fonction des retours de la soutenance\\
    \hline
\end{tabularx}

\section{Prévision des tâches pour la suite de l'itération par ordre de réalisation :}

\noindent\begin{tabularx}{17cm}{|p{2.5cm}|p{2.5cm}|p{1cm}|p{1.5cm}|p{2.5cm}|X|}
    \hline
    \textbf{Tâche} & \textbf{Nature} & \textbf{Qui} & \textbf{Temps} & \textbf{Réalisation} & \textbf{Commentaire} \\
    \hline
    Amélioration du CdC & production & ML, TT, RC & 2 h & Fait & (voirà la fin du tableau)\\
    \hline
    Sprint planning & concertation / production & ML, TTT, RC & 2 h & Indéterminé & Réalisation du sprint planning\\
    \hline
    Prise en main outils & Formation & ML, TTT, RC & X h & Fait & Maîtrise outils et libraries \\
    \hline
\end{tabularx} \\

Après notre soutenance, nous avons eu des retours concernant notre cahier des charges de la part de M. Guédon. Certains points était à garder, mais d'autres à revoir.\\
Les principaux points à revoir étaient :
\begin{itemize}
\item Ajout d'introductions / conclusions dans les sous-parties
\item Présence d'une table des figures
\item Ajout d'explications pour décrire le modèle du domaine
\item Comparaison entre les recherches
\item Etre encore plus précis sur le Gantt, la planification des itérations
\end{itemize}
Le professeur nous a autorisé à modifier notre cahier des charges afin de l'améliorer, et de le rendre d'ici jeudi 28 Novembre 2019.


\section{Nouvelles complémentaires}

\noindent\begin{tabularx}{17cm}{|p{5cm}|X|}
\hline
   \begin{minipage}{.3\textwidth}
      \includegraphics[width=\linewidth]{img/smhappy.jpg}
    \end{minipage}
    & 
    Les choses positives de la semaine :\\
\hline
     \begin{minipage}{.3\textwidth}
      \includegraphics[width=\linewidth]{img/smneutral.jpg}
    \end{minipage}
    &
    Les inquiètudes de la semaine : Revoir le cahier des charges, temps pour commencer la première itération\\
\hline
     \begin{minipage}{.3\textwidth}
      \includegraphics[width=\linewidth]{img/smsad.jpg}
    \end{minipage}
    &
    Les problèmes importants de la semaine :\\
\hline
\end{tabularx}

\end{document}
