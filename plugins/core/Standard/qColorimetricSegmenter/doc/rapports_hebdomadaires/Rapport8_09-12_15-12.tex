\documentclass[12pt,titlepage,french]{article}
\usepackage{babel}
\usepackage{graphicx}
\usepackage[margin=2.5cm]{geometry}

\usepackage[hidelinks]{hyperref}
\usepackage{tabularx}
\usepackage[utf8]{inputenc}
\usepackage[T1]{fontenc}
\pagestyle{plain}

\usepackage{booktabs,makecell,tabu}
\renewcommand\theadfont{\bfseries}

\linespread{1.5}

\begin{document}
%\renewcommand{\thesection}{\arabic{section}} % utilisé pour spécifier la numérotation des sections

\begin{titlepage}
\newcommand{\HRule}{\rule{\linewidth}{0.5mm}}
\center

  \includegraphics[width=0.45\textwidth]{../ressources/img_logos/logo_polytech.png}\\[1cm]
   
  \includegraphics[width=0.45\textwidth]{../ressources/img_logos/logo_taglabs.png}


\HRule \\[0.4cm]
{ \huge \bfseries Rapport hebdomadaire\\[0.15cm] }
Classification colorimétrique de nuages de points 3D\\
Semaine du 09/12 au 15/12
\HRule \\[1.5cm]
Ronan Collier,
Mathieu Letrone,
Tri-Thien Truong
\\[1cm]
\end{titlepage}

\section{Travail effectué}

\noindent\begin{tabularx}{17cm}{|p{2.5cm}|p{2.5cm}|p{1cm}|p{1.5cm}|p{2.5cm}|X|}
    \hline
    \textbf{Tâche} & \textbf{Nature} & \textbf{Qui} & \textbf{Temps} & \textbf{Réalisation} & \textbf{Commentaire} \\
    \hline
    Comparaison et tests des bibliothèques en C++ & Production & TTT & 6 h & Terminé & Réalisation d'un document pour comparer les bibliothèques liées aux nuages de points, et rélisation de tests de ces dernières.\\
    \hline
    Réflexion sur l'architecture du code &  & TTT, RC, ML & 2 h & Terminé & \\
    \hline
    Rédaction rapport d'itération & Production & TTT, RC, ML & 4 h & En cours & \\
    \hline
\end{tabularx} \\

\section{Prévision des tâches pour la suite de l'itération par ordre de réalisation :}

\noindent\begin{tabularx}{17cm}{|p{2.5cm}|p{2.5cm}|p{1cm}|p{1.5cm}|p{2.5cm}|X|}
    \hline
    \textbf{Tâche} & \textbf{Nature} & \textbf{Qui} & \textbf{Temps} & \textbf{Réalisation} & \textbf{Commentaire} \\
    \hline
    Choix du langage & Recherche & TT & 4 h & Début & Réfléchir aux choix du développement en C++ ou Python, ou autre\\
    \hline
    Commencer la présentation pour le sprint review & Production & ML, TTT, RC & 3 h & Début  &  \\
    \hline
    Terminer le rapport d'itération & Production & ML, TTT, RC & 5h & A terminer & \\
    \hline
\end{tabularx} \\

\section{Nouvelles complémentaires}

\noindent\begin{tabularx}{17cm}{|p{5cm}|X|}
\hline
   \begin{minipage}{.3\textwidth}
      \includegraphics[width=\linewidth]{img/smhappy.jpg}
    \end{minipage}
    & 
    Les choses positives de la semaine : Nous avons pu bien avancer dans nos recherches, nous avons commencé à mettre au propre l'ensemble de nos résultats pour le rapport d'itération.\\
\hline
     \begin{minipage}{.3\textwidth}
      \includegraphics[width=\linewidth]{img/smneutral.jpg}
    \end{minipage}
    &
    Les inquiètudes de la semaine : \\
\hline
     \begin{minipage}{.3\textwidth}
      \includegraphics[width=\linewidth]{img/smsad.jpg}
    \end{minipage}
    &
    Les problèmes importants de la semaine :\\
\hline
\end{tabularx}

\end{document}
